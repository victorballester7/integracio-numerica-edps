\documentclass[10pt,a4paper]{article}
\usepackage[utf8]{inputenc}
\usepackage{amsthm, amsmath, mathtools, amssymb}
\usepackage[left=2cm,right=2cm,top=2cm,bottom=2cm]{geometry}
\usepackage[colorlinks,linkcolor=blue,citecolor=blue,urlcolor=blue]{hyperref}
\usepackage[catalan]{babel}
\usepackage{titlesec}
\usepackage{enumitem}
\usepackage{physics}
\usepackage{fancyhdr}
\usepackage{subcaption}

\newcommand{\NN}{\ensuremath{\mathbb{N}}} % set of natural numbers
\newcommand{\ZZ}{\ensuremath{\mathbb{Z}}} % set of integers
\newcommand{\QQ}{\ensuremath{\mathbb{Q}}} % set of rationals
\newcommand{\RR}{\ensuremath{\mathbb{R}}} % set of real numbers
\newcommand{\CC}{\ensuremath{\mathbb{C}}} % set of complex numbers
\newcommand{\KK}{\ensuremath{\mathbb{K}}} % a general field

\newcommand{\vf}[1]{\boldsymbol{\mathrm{#1}}} % math style for vectors and matrices and vector-values functions (previously it was \*vb{#1} but this does not apply to greek letters)
\newcommand{\ii}{\mathrm{i}} % imaginary unit
\renewcommand{\O}{\mathrm{O}} % big O-notation

\newtheorem{theorem}{Teorema}
\newtheorem{exercici}{Exercice}
\newtheorem{prop}{Proposició}
\theoremstyle{definition}
\newtheorem{definition}{Definició}
\theoremstyle{remark}
\newtheorem*{res}{Resolution}
\DeclareDocumentCommand\derivative{ s o m g d() }{ 
  % Total derivative
  % s: star for \flatfrac flat derivative
  % o: optional n for nth derivative
  % m: mandatory (x in df/dx)
  % g: optional (f in df/dx)
  % d: long-form d/dx(...)
    \IfBooleanTF{#1}
    {\let\fractype\flatfrac}
    {\let\fractype\frac}
    \IfNoValueTF{#4}
    {
        \IfNoValueTF{#5}
        {\fractype{\diffd \IfNoValueTF{#2}{}{^{#2}}}{\diffd #3\IfNoValueTF{#2}{}{^{#2}}}}
        {\fractype{\diffd \IfNoValueTF{#2}{}{^{#2}}}{\diffd #3\IfNoValueTF{#2}{}{^{#2}}} \argopen(#5\argclose)}
    }
    {\fractype{\diffd \IfNoValueTF{#2}{}{^{#2}} #3}{\diffd #4\IfNoValueTF{#2}{}{^{#2}}}\IfValueT{#5}{(#5)}}
} % differential operator
\DeclareDocumentCommand\partialderivative{ s o m g d() }{ 
  % Total derivative
  % s: star for \flatfrac flat derivative
  % o: optional n for nth derivative
  % m: mandatory (x in df/dx)
  % g: optional (f in df/dx)
  % d: long-form d/dx(...)
  \IfBooleanTF{#1}
    {\let\fractype\flatfrac}
    {\let\fractype\frac}
    \IfNoValueTF{#4}{
      \IfNoValueTF{#5}
      {\fractype{\partial \IfNoValueTF{#2}{}{^{#2}}}{\partial #3\IfNoValueTF{#2}{}{^{#2}}}}
      {\fractype{\partial \IfNoValueTF{#2}{}{^{#2}}}{\partial #3\IfNoValueTF{#2}{}{^{#2}}} \argopen(#5\argclose)}
    }
    {\fractype{\partial \IfNoValueTF{#2}{}{^{#2}} #3}{\partial #4\IfNoValueTF{#2}{}{^{#2}}}\IfValueT{#5}{(#5)}}
} % partial differential operator

\titleformat{\section}
  {\normalfont\fontsize{11}{15}\bfseries}{\thesection}{1em}{}

% \renewcommand{\theenumi}{\textbf{\arabic{enumi}}}
\renewcommand{\theenumi}{\alph{enumi}}
\renewcommand{\theenumiii}{\roman{enumiii}}
\renewcommand{\exp}[1]{\mathrm{e}^{#1}} % exponential function
\DeclareMathOperator*{\im}{Im}
\setlength\multlinegap{0pt} % disable the margins on \begin{multline} command.

\title{\bfseries\Large Exercise 19}

\author{Víctor Ballester Ribó\\NIU: 1570866}
\date{\parbox{\linewidth}{\centering
  Integració numèrica d'equacions en derivades parcials\endgraf
  Grau en Matemàtiques\endgraf
  Universitat Autònoma de Barcelona\endgraf
  Febrer de 2023}}
  \pagestyle{fancy}
  \fancyhf{}
  \fancyhfoffset[L]{1cm}
  \fancyhfoffset[R]{1cm}
  \rhead{NIU: 1570866}
  \lhead{Víctor Ballester}
  \cfoot{\thepage}
  %\setlength{\headheight}{13.6pt}

\setlength{\parindent}{0pt}
\begin{document}
\selectlanguage{catalan}
\maketitle
\begin{exercici}
  Show that the box scheme
  \begin{equation}\label{eq:scheme}
    \frac{1}{2k}\left[v_m^{n+1}+v_{m+1}^{n+1}-v_m^n-v_{m+1}^n\right] +\frac{a}{2h}\left[v_{m+1}^{n+1}-v_m^{n+1}+v_{m+1}^n-v_m^n\right]=0
  \end{equation}
  for the homogeneous one-way wave equation $u_t+au_x=0$ is accurate of order [2,3].
\end{exercici}
\begin{res}
  We need to check that:
  \begin{equation}
    \abs{\frac{\exp{kq(\xi)}-g(h\xi)}{k}}\leq C h^r\abs{\xi}^\rho
  \end{equation}
  for some constant $C\in\RR$ and with $r=2$ and $\rho=3$. Here $g$ is the amplification factor and $q$ is such that:
  $$
    \widehat{u}_t=q(\xi)\widehat{u}
  $$

  Let's compute first the amplification factor $g$. Let $v_m^n=g^n\exp{\ii m\theta}$. Then, substituting this into \eqref{eq:scheme} and using that $\lambda = k/h$, we get the following equation:
  $$
    \left[g+g\exp{\ii \theta}-1-\exp{\ii\theta}\right]+ a\lambda \left[g\exp{\ii\theta}-g+\exp{\ii\theta}-1\right]=0
  $$
  Factoring this equation we have:
  \begin{align*}
    (g-1)(1+\exp{\ii\theta})                            & = -a\lambda (g+1)(\exp{\ii\theta} - 1)                                \\
    (g-1)\frac{\exp{\ii\theta/2}+\exp{-\ii\theta/2}}{2} & = -a\lambda\ii (g+1)\frac{\exp{\ii\theta/2}-\exp{-\ii\theta/2}}{2\ii} \\
    (g-1)\cos(\theta/2)                                 & = -a\lambda\ii (g+1)\sin(\theta/2)                                    \\
    g - 1                                               & = -a\lambda\ii g\tan(\theta/2) - a\lambda\ii \tan(\theta/2)           \\
    g                                                   & = \frac{1-a\lambda\ii\tan(\theta/2)}{1+a\lambda\ii\tan(\theta/2)}
  \end{align*}
  Now let's find $q$. Taking the Fourier transform (on the $x$ variable) of the equation $u_t+au_x=0$ we have:
  $$
    \widehat{u}_t =\widehat{u_t}= - a\widehat{u_x}= -a\ii\xi\widehat{u}
  $$
  where the last equality follows from the identity $\widehat{u_x}(\xi)=\ii\xi\widehat{u}(\xi)$. So $q(\xi)=-a\ii\xi$.
  Let's study first the Taylor expansion of $g(h\xi)$. Recall that $\tan(x)=x+ \frac{x^3}{3}+\O(x^5)$, so:
  \begin{align*}
    1 - a\lambda\ii\tan(x)                              & = 1 - a\lambda \ii x - \frac{a\lambda\ii}{3}x^3 + \O(x^5)                                                                                                                 \\
    \frac{1}{1+ a\lambda\ii\tan(x)}                     & = 1 - (a\lambda\ii\tan(x)) + {(a\lambda\ii\tan(x))}^2 -{(a\lambda\ii\tan(x))}^3 + \cdots                                                                                  \\
                                                        & = 1 - a\lambda\ii x -a^2\lambda^2x^2 + a\lambda\ii\frac{3a^2\lambda^2 - 1}{3}x^3+ \O(x^4)                                                                                 \\
    \frac{1- a\lambda\ii\tan(x)}{1+ a\lambda\ii\tan(x)} & = \left[1 - a\lambda\ii x - \frac{a\lambda\ii}{3}x^3 + \O(x^5)\right]\left[1 - a\lambda\ii x -a^2\lambda^2x^2 + a\lambda\ii\frac{3a^2\lambda^2 - 1}{3}x^3+ \O(x^4)\right] \\
                                                        & = 1 -2a\lambda \ii x -2a^2\lambda^2x^2 + 2a\lambda\ii\frac{3a^2\lambda^2-1}{3}x^3 +\O(x^4)
  \end{align*}
  Substituting $x=h\xi/2$ in the latter expression we have:
  $$
    g(h\xi) =  1 -a\lambda \ii h\xi -\frac{a^2\lambda^2}{2}h^2\xi^2 + a\lambda\ii\frac{3a^2\lambda^2-1}{12}h^3\xi^3 +\O(h^4)
  $$
  On the other hand the expansion of $\exp{kq(\xi)}=\exp{-a\lambda\ii h\xi}$ is:
  $$
    \exp{-a\lambda\ii h\xi}=1-a\lambda\ii h\xi-\frac{a^2\lambda^2}{2}h^2\xi^2+\frac{a^3\lambda^3\ii}{6} h^3\xi^3+\O(h^4)
  $$
  Thus:
  $$
    \abs{\frac{\exp{kq(\xi)}-g(h\xi)}{k}}=\abs{\frac{\frac{1}{12}a\lambda \ii (1-a^2\lambda^2)h^3\xi^3 + \O(h^4)}{\lambda h}}\leq  \frac{1}{12}\abs{a} \abs{1-a^2\lambda^2}h^2\abs{\xi}^3 +\O(h^3)
  $$
  So we take $C:=\frac{1}{12}\abs{a} \abs{1-a^2\lambda^2}$.
  % We need to show that for $k=\lambda h$ we have:
  % $$
  % P_{k,h}\phi-P\phi=P_{k,h}\phi=\O(k^2)+\O(h^3)
  % $$
  % because in this case we have $P\phi=\phi_t+a\phi_x=0$. Here $P_{k,h}$ is the operator of the box scheme. We have that:
  % \begin{multline*}
  % P_{k,h}\phi=\frac{1}{2k}\left[\phi(t+k,x)+\phi(t+k,x+h) - \phi(t,x)-\phi(t,x+h)\right] +\\+\frac{a}{2h}\left[\phi(t+k,x+h)-\phi(t+k,x)+\phi(t,x+h)-\phi(t,x)\right]
  % \end{multline*}
  % Now, Taylor-expanding the equation in $k$ and $h$ centered at $\phi=\phi(t,x)$, and omitting the evaluation of the functions at $(t,x)$ to simplify the reading, we have:
  % \begin{align*}
  % P_{k,h}\phi=\frac{1}{2k}\bigg[ & \phi+ k\phi_t + \frac{k^2}{2}\phi_{tt}+\frac{k^3}{6}\phi_{ttt}+                                                                                                                              \\
  %  & + \phi +k\phi_t + h\phi_x + \frac{k^2}{2}\phi_{tt} + \frac{h^2}{2}\phi_{xx} + kh\phi_{tx} +\frac{k^3}{6}\phi_{ttt}+\frac{k^2h}{2}\phi_{ttx}+\frac{kh^2}{2}\phi_{txx}+\frac{h^3}{6}\phi_{xxx} \\
  %  & -\phi                                                                                                                                                                                        \\
  %  & -\phi-h\phi_x-\frac{h^2}{2}\phi_{xx}-\frac{h^3}{6}\phi_{xxx}                                                                                                                                 \\
  %  & +\O(k^4)+\O(k^3h)+\O(k^2h^2)+(kh^3)\bigg]                                                                                                                                                    \\
  % +\frac{a}{2h}\bigg[            & \phi + k\phi_t + h\phi_x + \frac{k^2}{2}\phi_{tt}+\frac{h^2}{2}\phi_{xx} + kh\phi_{tx} +\frac{k^3}{6}\phi_{ttt}+\frac{k^2h}{2}\phi_{ttx}+\frac{kh^2}{2}\phi_{txx}+\frac{h^3}{6}\phi_{xxx}    \\
  %  & -\phi-k\phi_t-\frac{k^2}{2}\phi_{tt}-\frac{k^3}{6}\phi_{ttt}                                                                                                                                 \\
  %  & +\phi+h\phi_x+\frac{h^2}{2}\phi_{xx}+\frac{h^3}{6}\phi_{xxx}                                                                                                                                 \\
  %  & -\phi                                                                                                                                                                                        \\
  %  & +\O(k^3h)+\O(k^2h^2)+(kh^3)+\O(h^4)\bigg]
  % \end{align*}
  % Note that all the terms of $\O(h^\alpha)$ in the first summand vanish and the same applies for $\O(k^\alpha)$ in the second summand. This latter expression simplifies to:
  % \begin{align*}
  % P_{k,h}\phi= & \ \phi_t+\frac{k}{2}\phi_{tt}+ \frac{k^2}{6}\phi_{ttt}+ \frac{h}{2}\phi_{tx}+\frac{kh}{4}\phi_{ttx}+\frac{h^2}{4}\phi_{txx}              \\
  %  & +a\left(\phi_x+\frac{h}{2}\phi_{xx}+ \frac{h^2}{6}\phi_{xxx}+ \frac{k}{2}\phi_{tx}+\frac{k^2}{4}\phi_{ttx}+\frac{kh}{4}\phi_{txx}\right) \\
  %  & +\O{(k^3)}+\O(k^2h)+\O(kh^2)+\O{(h^3)}
  % \end{align*}
  % Now using that $\phi_t+a\phi_x=0$, and so $\phi_{tt}+a\phi_{tx}=0$, $\phi_{tx}+a\phi_{xx}=0$ and $\phi_{ttx}+a\phi_{txx}=0$, we have:
  % \begin{equation*}
  % P_{k,h}\phi=\frac{k^2}{6}\phi_{ttt}+\frac{h^2}{4}\phi_{txx} +a\frac{h^2}{6}\phi_{xxx}+ a\frac{k^2}{4}\phi_{ttx}+\O(k^3)+\O(k^2h)+\O(kh^2)+\O(h^3)
  % \end{equation*}
  % Now using that $a\phi_{xxx}=-\phi_{txx}$ and $a\phi_{ttx}=-\phi_{ttt}$, we have:
  % \begin{equation*}
  % P_{k,h}\phi=-\frac{k^2}{12}\phi_{ttt}+\frac{h^2}{12}\phi_{txx} +\O(k^3)+\O(k^2h)+\O(kh^2)+\O(h^3)
  % \end{equation*}
  % Finally, letting $k=\lambda h$ we have:
  % \begin{equation*}
  % P_{k,h}\phi=-\frac{k^2}{12}\phi_{ttt}+\frac{k^2}{12\lambda^2}\phi_{txx} +\O(h^3)=\O(k^2)+\O(h^3)
  % \end{equation*}
\end{res}
\end{document}
\documentclass[10pt,a4paper]{article}
\usepackage[utf8]{inputenc}
\usepackage{amsthm, amsmath, mathtools, amssymb}
\usepackage[left=2cm,right=2cm,top=2cm,bottom=2cm]{geometry}
\usepackage[colorlinks,linkcolor=blue,citecolor=blue,urlcolor=blue]{hyperref}
\usepackage[catalan]{babel}
\usepackage{titlesec}
\usepackage{enumitem}
\usepackage{physics}
\usepackage{fancyhdr}
\usepackage{subcaption}
\usepackage{parskip}

\newcommand{\NN}{\ensuremath{\mathbb{N}}} % set of natural numbers
\newcommand{\ZZ}{\ensuremath{\mathbb{Z}}} % set of integers
\newcommand{\QQ}{\ensuremath{\mathbb{Q}}} % set of rationals
\newcommand{\RR}{\ensuremath{\mathbb{R}}} % set of real numbers
\newcommand{\CC}{\ensuremath{\mathbb{C}}} % set of complex numbers
\newcommand{\KK}{\ensuremath{\mathbb{K}}} % a general field

\newcommand{\vf}[1]{\boldsymbol{\mathrm{#1}}} % math style for vectors and matrices and vector-values functions (previously it was \*vb{#1} but this does not apply to greek letters)
\newcommand{\ii}{\mathrm{i}} % imaginary unit
\renewcommand{\O}[1]{\mathrm{O}\left(#1\right)} % big O-notation

\makeatletter
\setlength{\@fptop}{0pt}
\makeatother


\newtheorem{theorem}{Teorema}
\newtheorem{exercici}{Exercice}
\newtheorem{prop}{Proposició}
\theoremstyle{definition}
\newtheorem{definition}{Definició}
\theoremstyle{remark}
\newtheorem*{res}{Resolution}
\DeclareDocumentCommand\derivative{ s o m g d() }{ 
  % Total derivative
  % s: star for \flatfrac flat derivative
  % o: optional n for nth derivative
  % m: mandatory (x in df/dx)
  % g: optional (f in df/dx)
  % d: long-form d/dx(...)
    \IfBooleanTF{#1}
    {\let\fractype\flatfrac}
    {\let\fractype\frac}
    \IfNoValueTF{#4}
    {
        \IfNoValueTF{#5}
        {\fractype{\diffd \IfNoValueTF{#2}{}{^{#2}}}{\diffd #3\IfNoValueTF{#2}{}{^{#2}}}}
        {\fractype{\diffd \IfNoValueTF{#2}{}{^{#2}}}{\diffd #3\IfNoValueTF{#2}{}{^{#2}}} \argopen(#5\argclose)}
    }
    {\fractype{\diffd \IfNoValueTF{#2}{}{^{#2}} #3}{\diffd #4\IfNoValueTF{#2}{}{^{#2}}}\IfValueT{#5}{(#5)}}
} % differential operator
\DeclareDocumentCommand\partialderivative{ s o m g d() }{ 
  % Total derivative
  % s: star for \flatfrac flat derivative
  % o: optional n for nth derivative
  % m: mandatory (x in df/dx)
  % g: optional (f in df/dx)
  % d: long-form d/dx(...)
  \IfBooleanTF{#1}
    {\let\fractype\flatfrac}
    {\let\fractype\frac}
    \IfNoValueTF{#4}{
      \IfNoValueTF{#5}
      {\fractype{\partial \IfNoValueTF{#2}{}{^{#2}}}{\partial #3\IfNoValueTF{#2}{}{^{#2}}}}
      {\fractype{\partial \IfNoValueTF{#2}{}{^{#2}}}{\partial #3\IfNoValueTF{#2}{}{^{#2}}} \argopen(#5\argclose)}
    }
    {\fractype{\partial \IfNoValueTF{#2}{}{^{#2}} #3}{\partial #4\IfNoValueTF{#2}{}{^{#2}}}\IfValueT{#5}{(#5)}}
} % partial differential operator

\titleformat{\section}
  {\normalfont\fontsize{11}{15}\bfseries}{\thesection}{1em}{}

% \renewcommand{\theenumi}{\textbf{\arabic{enumi}}}
\renewcommand{\theenumi}{\alph{enumi}}
\renewcommand{\theenumiii}{\roman{enumiii}}
\renewcommand{\exp}[1]{\mathrm{e}^{#1}} % exponential function
\DeclareMathOperator*{\im}{Im}
\setlength\multlinegap{0pt} % disable the margins on \begin{multline} command.

\title{\bfseries\Large Diferències finites per a equacions d'evolució}

\author{Víctor Ballester Ribó\\NIU: 1570866}
\date{\parbox{\linewidth}{\centering
  Integració numèrica d'equacions en derivades parcials\endgraf
  Grau en Matemàtiques\endgraf
  Universitat Autònoma de Barcelona\endgraf
  Juny de 2023}}
  \pagestyle{fancy}
  \fancyhf{}
  \fancyhfoffset[L]{1cm}
  \fancyhfoffset[R]{1cm}
  \rhead{NIU: 1570866}
  \lhead{Víctor Ballester}
  \cfoot{\thepage}
  %\setlength{\headheight}{13.6pt}

\setlength{\parindent}{0pt}
\begin{document}
\selectlanguage{catalan}
\maketitle
L'objectiu d'aquesta pràctica és resoldre el problema mixt següent per a l'equació de la calor en una placa rectangular:
\begin{equation}\label{eq:problema}
  \begin{cases}
    u_t-\Delta u = f & \text{a } [0,T]\times\Omega         \\
    u=g              & \text{a } [0,T]\times\partial\Omega \\
    u=h              & \text{a } \{0\}\times \Omega
  \end{cases}
\end{equation}
on $\Omega$ és la regió rectangular de la placa, que en el nostre cas serà $\Omega = [0,1]\times[0,1]$, i $T$ és el temps final de la simulació.

Donades funcions $f(t,x,y)$, $g(t,x,y)$ i $h(x,y)$ resoldrem aquest problema usant diferències finites. Discretitzem primer l'espai i el temps. Considerem dues particions de l'interval $[0,1]$ en $n_x+1$ i $n_y+1$ punts equidistants amb separacions $\delta_x=1/n_x$ i $\delta_y=1/n_y$ respectivament. A més, discretitzem l'interval temporal $[0,T]$ en $n_t+1$ punts equidistants amb separació $\delta_t=T/n_t$. D'ara endavant, denotarem la solució $u(k\delta_t, i\delta_x, j\delta_y)$ per $u_{k,i,j}$ i les seves respectives aproximacions per diferències finites per $v_{k,i,j}$.

Considerem primer el següent mètode explícit, que s'obté de prendre diferències centrades de segon ordre per l'espai i de primer ordre endavant per al temps:
$$
  v_{k+1,i,j} = (1-2\mu_x - 2\mu_y) v_{k,i,j} + \mu_x (v_{k,i+1,j} + v_{k,i-1,j}) + \mu_y (v_{k,i,j+1} + v_{k,i,j-1}) + \delta_t f_{k,i,j}
$$
on $\mu_x = \delta_t/{\delta_x}^2$ i $\mu_y = \delta_t/{\delta_y}^2$ i $f_{k,i,j}=f(k\delta_t, i\delta_x, j\delta_y)$. Aquest mètode és sempre consistent i a més és estable si $1- 2\mu_x -2 \mu_y \geq 0$. A més, quan convergeix, l'error de convergència és $\O{\delta_t + \delta_x^2 + \delta_y^2}$.

Fixem-nos que aquest mètode és exacte per a polinomis de grau $\leq 1$ en $t$ i grau $\leq 3$ en l'espai. Això es deu a l'expansió en sèrie de Taylor de $u$: d'una banda el coeficient que multiplica l'error $\O{\delta_t}$ de la diferència finita pel temps de primer ordre involucra $u_{tt}=0$ i anàlogament amb les derivades per l'espai d'ordre 2 (el terme de l'error involucra derivades d'ordre 4, que són nu\lgem es). Això ho podem comprovar també numèricament. Considerem el polinomi
$$u(t,x,y)=1+t+x^2+y^2+x^3$$
Prenent $f(t,x,y)= -3-6x$, $g(t,x,y)=1+t+x^2+y^2+x^3$ i $h(x,y)=1+x^2+y^2+x^3$ tenim que la solució del problema \eqref{eq:problema} és precisament $u(t,x,y)$. Així doncs, si prenem per exemple $\delta_x=0.1$, $\delta_y=0.1$ i $\delta_t=0.001$, que compleixen la condició d'estabilitat, per a $\Omega = [0,1]\times[0,1]$ i $T=1$ obtenim un error en norma $\norm{\cdot}_\infty$ entre la solució real i l'aproximada de $5.329\,07\cdot 10^{-15}$, que es pot considerar com a error d'aritmètica en punt flotant.\vspace*{0.25cm}

Considerem ara una placa d'acer de densitat $\rho= 7.8\ \frac{\mathrm{g}}{\mathrm{cm}^3}$, conductivitat tèrmica $\kappa = 0.13\ \frac{\mathrm{cal}}{\mathrm{cm}\cdot\mathrm{s}\cdot^\circ\mathrm{C}}$ i calor específic $c=0.11\ \frac{\mathrm{cal}}{\mathrm{g}\cdot^\circ\mathrm{C}}$. Posem una font de calor a sota de la placa que genera $F(\tau, x,y) \ \frac{\mathrm{cal}}{\mathrm{s}\cdot\mathrm{cm}^3}$ a l'instant $\tau$ i al punt $(x,y)$, on $F$ està definida per:
$$
  F(\tau, x,y) = \begin{cases}
    100 & \text{si } \norm{(x,y)-(\frac{1}{2},\frac{1}{2})}_2 <\frac{1}{5} \\
    0   & \text{altrament}
  \end{cases}
$$
Suposem també que en l'instant $\tau=0$, la placa està a temperatura $0\ ^\circ\mathrm{C}$ i per a tot temps $\tau$ la frontera de la placa es manté també a $0\ ^\circ\mathrm{C}$. En aquestes condicions, l'evolució de la temperatura sobre la placa ve donada per la solució del següent problema mixt:
$$
  \begin{cases}
    c\rho v_{\tau} -\kappa \Delta v = F(\tau,x,y) & \text{a } [0,T] \times \Omega         \\
    v=0                                           & \text{a } [0,T] \times \partial\Omega \\
    v=0                                           & \text{a } \{0\} \times \Omega         \\
  \end{cases}
$$
Usant el canvi de variables $t=\frac{\kappa}{c\rho}\tau$, tenim que $\dv{t}{\tau}= \frac{\kappa}{c\rho}$ i per tant si $u(t,x,y)=v(\tau,x,y)$, usant la regla de la cadena, $u_t = v_\tau \frac{c\rho}{\kappa}$ (i clarament $\Delta u=\Delta v$). Així doncs, el problema anterior es pot reescriure com:
$$
  \begin{cases}
    \displaystyle u_t -\Delta u = \frac{1}{\kappa} F\left(\frac{c\rho}{\kappa}t,x,y\right) & \displaystyle\text{a } \left[0,\frac{\kappa}{c\rho}T\right] \times \Omega  \vspace{0.1cm} \\
    u=0                                                                                    & \displaystyle\text{a } \left[0,\frac{\kappa}{c\rho}T\right] \times \partial\Omega         \\
    u=0                                                                                    & \text{a } \{0\} \times \Omega                                                             \\
  \end{cases}
$$

La condició d'estabilitat en l'esquema explícit que hem descrit anteriorment és necessària. Fem un experiment numèric per veure què podem obtenir si no se satisfà aquesta condició. Per fer-ho prenem $\delta_x=\delta_y=\delta_t=0.1$ i $T=1$. Amb aquests paràmetres obtenim que en $\tau=0.9$, la temperatura en la regió de la placa amb $x=0.3$ és:
\begin{table}[ht]
  \centering
  \begin{tabular}{|c|c|c|}
    \hline
    $x$ & $y$ & $v(0.9,x,y)$              \\ \hline\hline
    0.3 & 0.0 & 0                         \\
    0.3 & 0.1 & $42\,286\,728.810\,146$   \\
    0.3 & 0.2 & $-111\,988\,126.131\,011$ \\
    0.3 & 0.3 & $199\,241\,046.679\,864$  \\
    0.3 & 0.4 & $-259\,680\,534.657\,868$ \\
    0.3 & 0.5 & $263\,234\,236.257\,432$  \\
    0.3 & 0.6 & $-210\,322\,365.754\,957$ \\
    0.3 & 0.7 & $126\,199\,592.465\,624$  \\
    0.3 & 0.8 & $-54\,636\,211.331\,768$  \\
    0.3 & 0.9 & $16\,318\,289.975\,168$   \\
    0.3 & 1.0 & 0                         \\
    \hline
  \end{tabular}
\end{table}

Observem valors molt grossos i negatius, que no tenen sentit físic, ja que estan fora de l'interval $[0,100]$, que és l'interval per on s'haurien de moure les temperatures. El problema d'aquest mètode explícit és que requereix d'un cost computacional elevat quan volem els resultats amb una precisió raonable. Per exemple, si volem que $\O{\delta_t + {\delta_x}^2 + {\delta_y}^2} = \O{10^{-4}}$ aleshores podem prendre $\delta_x= \delta_y=0.01$ i $\delta_t=10^{-5}$. D'aquesta manera, $1-2\mu_x-2\mu_y = 0.6>0$ i, per tant, el mètode és estable. Ara bé, per arribar fins a $\tau=1$ necessitem emprar molt de temps. A la taula següent mostrem el temps que triga el programa a calcular els valors de la temperatura en els punts de la placa fins a diferents temps finals:
\begin{table}[ht]
  \centering
  \begin{tabular}{|c|c|}
    \hline
    $T$    & Temps emprat [s] \\ \hline\hline
    0.0001 & $0.047\,858$     \\
    0.001  & $0.569\,544$     \\
    0.01   & $5.613\,888$     \\
    0.1    & $57.489\,433$    \\
    \hline
  \end{tabular}
  \caption{Temps que triga l'esquema explícit en calcular i escriure a un fitxer tots els valors en cada graella (usant $\delta_x= \delta_y=0.01$ i $\delta_t=10^{-5}$) fins a diferents temps finals.}
\end{table}

Aproximadament cada temps es multiplica per 10 respecte al temps anterior. Per tant, calculem que trigaríem de l'ordre de 10 minuts per a calcular els valors de la temperatura fins a $\tau=1$.\vspace{0.25cm}

Una solució alternativa és emprar un mètode implícit, com ara el de Crank-Nicolson:
\begin{align*}
  \frac{v_{k+1,i,j}-v_{k,i,j}}{\delta_t} & - \frac{1}{2}\left(\frac{v_{k+1,i+1,j}-2v_{k+1,i,j}+v_{k+1,i-1,j}}{{\delta_x}^2} +\frac{v_{k,i+1,j}-2v_{k,i,j}+v_{k,i-1,j}}{{\delta_x}^2} \right)  -                                             \\
                                         & -\frac{1}{2}\left( \frac{v_{k+1,i,j+1}-2v_{k+1,i,j}+v_{k+1,i,j-1}}{{\delta_y}^2}+ \frac{v_{k,i,j+1}-2v_{k,i,j}+v_{k,i,j-1}}{{\delta_y}^2}\right) = \frac{1}{2}\left(f_{k+1,i,j}+f_{k,i,j}\right)
\end{align*}
on hem usat la notació de \eqref{eq:problema}. Per resoldre aquest sistema, hem de resoldre un sistema lineal en les incògnites $\{v_{k+1,i,j}\}_{i,j}$, i usarem el mètode SOR amb paràmetre $\omega = 1.7$ per a fer-ho.

Ara, però, a diferència del que teníem amb el mètode explícit, l'esquema de diferències finites no serà exacte quan l'apliquem a polinomis. En efecte, prenent els mateixos valors que anteriorment ($\delta_x=\delta_y=0.1$, $\delta_t=0.001$ i $T=1$), obtenim un error en norma $\norm{\cdot}_\infty$ de $9.17816\times 10^{-5}$, que està lluny de ser considerat com a error de punt flotant.

Tot i això, aquest mètode ens permet obtenir una precisió molt més elevada en un temps raonable, ja que no tenim cap condició d'estabilitat a satisfer-se i a més l'ordre de convergència és $\O{{\delta_t}^2 + {\delta_x}^2 + {\delta_y}^2}$. A la taula següent mostrem el temps que triga el programa a calcular els valors de la temperatura en els punts de la placa fins a diferents temps finals si volem obtenir una precisió de l'ordre de $10^{-4}$:
\begin{table}[ht]
  \centering
  \begin{tabular}{|c|c|}
    \hline
    $T$  & Temps emprat [s] \\ \hline\hline
    0.01 & $0.107\,239$     \\
    0.1  & $1.056\,454$     \\
    1    & $9.937\,089$     \\
    \hline
  \end{tabular}
  \caption{Temps que triga l'esquema implícit de Crank-Nicolson en calcular i escriure a un fitxer tots els valors en cada graella (usant $\delta_x= \delta_y=0.01$ i $\delta_t=0.001$) fins a diferents temps finals.}
\end{table}

Finalment, donant aquest esquema com a bo, els resultats que obtenim en integrar l'equació de la calor sobre la placa d'acer es mostren a les següents imatges:
\begin{figure}[ht]
  \centering\vspace{-1cm}
  \begin{subfigure}{0.49\linewidth}
    \centering
    \includegraphics[width=\textwidth]{../plot/Images/heatmap_00010.pdf}
  \end{subfigure}\hfill
  \begin{subfigure}{0.49\linewidth}
    \centering
    \includegraphics[width=\textwidth]{../plot/Images/heatmap_00030.pdf}
  \end{subfigure}\\\vspace{-1cm}
  \begin{subfigure}{0.49\linewidth}
    \centering
    \includegraphics[width=\textwidth]{../plot/Images/heatmap_00060.pdf}
  \end{subfigure}\hfill
  \begin{subfigure}{0.49\linewidth}
    \centering
    \includegraphics[width=\textwidth]{../plot/Images/heatmap_00100.pdf}
  \end{subfigure}
  \caption{Temperatura a la placa d'acer en els temps $\tau=0.1,0.3,0.6,1$.}
\end{figure}

També podem representar diverses seccions de la placa i graficar les diverses corbes que obtenim en diversos temps:

\begin{figure}[t!]
  \centering
  \begin{subfigure}{0.49\linewidth}
    \centering
    \includegraphics[width=\linewidth]{Images/graphic50.pdf}
    \caption{Secció $x=0.5$}
  \end{subfigure}\hfill
  \begin{subfigure}{0.49\linewidth}
    \centering
    \includegraphics[width=\linewidth]{Images/graphic20.pdf}
    \caption{Secció $x=0.2$}
  \end{subfigure}
  \caption{Evolució de la temperatura a les seccions $x=0.5$ i $x=0.2$ de la placa.}
\end{figure}
\end{document}